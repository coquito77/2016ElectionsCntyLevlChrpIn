\documentclass[11pt,]{article}
\usepackage[]{mathpazo}
\usepackage{amssymb,amsmath}
\usepackage{ifxetex,ifluatex}
\usepackage{fixltx2e} % provides \textsubscript
\ifnum 0\ifxetex 1\fi\ifluatex 1\fi=0 % if pdftex
  \usepackage[T1]{fontenc}
  \usepackage[utf8]{inputenc}
\else % if luatex or xelatex
  \ifxetex
    \usepackage{mathspec}
  \else
    \usepackage{fontspec}
  \fi
  \defaultfontfeatures{Ligatures=TeX,Scale=MatchLowercase}
\fi
% use upquote if available, for straight quotes in verbatim environments
\IfFileExists{upquote.sty}{\usepackage{upquote}}{}
% use microtype if available
\IfFileExists{microtype.sty}{%
\usepackage{microtype}
\UseMicrotypeSet[protrusion]{basicmath} % disable protrusion for tt fonts
}{}
\usepackage[margin=1in]{geometry}
\usepackage{hyperref}
\hypersetup{unicode=true,
            pdftitle={A Pandoc Markdown Article Starter and Template},
            pdfkeywords={pandoc, r markdown, knitr},
            pdfborder={0 0 0},
            breaklinks=true}
\urlstyle{same}  % don't use monospace font for urls
\usepackage{graphicx,grffile}
\makeatletter
\def\maxwidth{\ifdim\Gin@nat@width>\linewidth\linewidth\else\Gin@nat@width\fi}
\def\maxheight{\ifdim\Gin@nat@height>\textheight\textheight\else\Gin@nat@height\fi}
\makeatother
% Scale images if necessary, so that they will not overflow the page
% margins by default, and it is still possible to overwrite the defaults
% using explicit options in \includegraphics[width, height, ...]{}
\setkeys{Gin}{width=\maxwidth,height=\maxheight,keepaspectratio}
\IfFileExists{parskip.sty}{%
\usepackage{parskip}
}{% else
\setlength{\parindent}{0pt}
\setlength{\parskip}{6pt plus 2pt minus 1pt}
}
\setlength{\emergencystretch}{3em}  % prevent overfull lines
\providecommand{\tightlist}{%
  \setlength{\itemsep}{0pt}\setlength{\parskip}{0pt}}
\setcounter{secnumdepth}{0}
% Redefines (sub)paragraphs to behave more like sections
\ifx\paragraph\undefined\else
\let\oldparagraph\paragraph
\renewcommand{\paragraph}[1]{\oldparagraph{#1}\mbox{}}
\fi
\ifx\subparagraph\undefined\else
\let\oldsubparagraph\subparagraph
\renewcommand{\subparagraph}[1]{\oldsubparagraph{#1}\mbox{}}
\fi

%%% Use protect on footnotes to avoid problems with footnotes in titles
\let\rmarkdownfootnote\footnote%
\def\footnote{\protect\rmarkdownfootnote}

%%% Change title format to be more compact
\usepackage{titling}

% Create subtitle command for use in maketitle
\newcommand{\subtitle}[1]{
  \posttitle{
    \begin{center}\large#1\end{center}
    }
}

\setlength{\droptitle}{-2em}
  \title{A Pandoc Markdown Article Starter and Template}
  \pretitle{\vspace{\droptitle}\centering\huge}
  \posttitle{\par}
  \author{true \\ true \\ true}
  \preauthor{\centering\large\emph}
  \postauthor{\par}
  \predate{\centering\large\emph}
  \postdate{\par}
  \date{January 19, 2018}

\usepackage{booktabs,float,longtable }
\hypersetup{breaklinks=true}
\usepackage{breakurl}

\begin{document}
\maketitle
\begin{abstract}
This document provides an introduction to R Markdown, argues for
its\ldots{}
\end{abstract}

\includegraphics{2016ElectionsCntyLevlChrpIn_files/figure-latex/plotcombo-1.pdf}

\includegraphics{2016ElectionsCntyLevlChrpIn_files/figure-latex/plotcombo2-1.pdf}

\% latex table generated in R 3.4.2 by xtable 1.8-2 package \% Wed Jan
17 20:49:20 2018

\begin{longtable}{rrrrrrrrrrr}
  \hline
 & A & B & C & D & E & F & G & H & I \footnote{my tricky footnote !!} & J \\ 
  \hline
1 & -0.84 & 1.04 & -0.27 & 0.73 & 0.32 & -0.63 & -0.17 & -0.52 & -0.71 & -0.93 \\ 
  2 & -1.24 & -1.18 & -1.11 & -0.66 & -0.48 & -0.55 & -1.52 & 1.52 & -0.69 & 0.56 \\ 
  3 & 0.45 & -0.54 & 0.77 & 1.96 & 1.01 & 0.57 & 0.50 & 2.02 & 1.10 & 0.02 \\ 
  4 & -0.24 & -0.08 & -0.48 & 0.50 & -0.12 & -0.51 & -0.46 & 0.69 & 0.47 & -0.18 \\ 
  5 & -0.46 & 1.22 & 1.60 & 0.80 & -1.23 & -1.49 & 0.96 & 0.87 & -0.44 & -0.19 \\ 
  6 & -0.54 & 0.64 & -0.84 & 2.47 & -1.41 & 0.11 & 0.73 & -0.86 & -1.10 & -0.73 \\ 
   \hline
\hline
\caption{Example of longtable} 
\label{tabbig}
\end{longtable}

I'm writing something here to test \footnote{footnotes working fine}
several features.

\begin{thebibliography}{9}

\bibitem{lamport94}
  Leslie Lamport,
  \textit{\LaTeX: a document preparation system},
  Addison Wesley, Massachusetts,
  2nd edition,
  1994.

\bibitem{latexcompanion} 
Michel Goossens, Frank Mittelbach, and Alexander Samarin. 
\textit{The \LaTeX\ Companion}. 
Addison-Wesley, Reading, Massachusetts, 1993.
 
\bibitem{einstein} 
Albert Einstein. 
\textit{Zur Elektrodynamik bewegter K{\"o}rper}. (German) 
[\textit{On the electrodynamics of moving bodies}]. 
Annalen der Physik, 322(10):891–921, 1905.
 
\bibitem{knuthwebsite} 
Knuth: Computers and Typesetting,
\url{http://www.rivcoems.org/Portals/0/Documents/DOCUMENTS/CONTRACTS/AMR/Current%20AMR%20Contract%20%202015.pdf}


\end{thebibliography}


\end{document}
